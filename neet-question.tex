\documentclass{article}
\usepackage[ utf8 ]{inputenc}
\usepackage[margin=1in]{geometry}
\usepackage{amsfonts}
\usepackage[english] {babel}
\usepackage{chemfig}
\usepackage[version=4]{mhchem}
\usepackage{circuitikz}

%Reference: https://www.vedantu.com/neet/neet-sample-question-paper-1
%-------------------------------------------------------------------------------------------

\begin{document}

\begin{center}
\huge
\textbf{NEET EXAM}\\
\Large \today
\vspace{1 cm}
\end{center}

\textbf{\underline{\large Max. Marks: 180} \hspace{10 cm} \underline{\large Duration: 3 Hrs}}
\\ \\ \\
\textbf{This paper consists of Chemistry, Physics and Biology \vspace{.5 cm}\\
Chemistry 
\begin{itemize}
	\item Multiple Choice Questions with one correct answer Question No. 1 to 45. A correct answer carries 1 Mark. A wrong answer carries a penalty of 0.25 marks.
\end{itemize}
Physics
\begin{itemize}
	\item Multiple Choice Questions with one correct answer Question No. 46 to 90. A correct answer carries 1 Mark. A wrong answer carries a penalty of 0.25 marks.
\end{itemize}
Biology
\begin{itemize}
	\item Multiple Choice Questions with one correct answer Question No. 91 to 180. A correct answer carries 1 Mark. A wrong answer carries a penalty of 0.25 marks.
\end{itemize}
}
\vspace{0.6 cm}
\begin{center}
\textbf{\underline{Useful Data}}\\
\end{center}
\texttt{Atomic Wt.:}\\
N = 14; O = 16; H = 1; S = 32; Cl = 35.5; Mn = 55; Na = 23; C = 12; Ag = 108; K =39; Fe = 56; \\Pb = 207\\ \\
\texttt{Physical constants:}\\
$h = 6.626 \times 10^{-34}$ J.sec,  $N_A = 6.022 \times 10^{23} mol^{-1}$, $C = 2.996 \times 10^9 ms^{-1}$, $m_e = 9.1 \times 10^{-31}Kg$\\

\vspace{1 cm}
\begin{center}
 \textbf{\large \underline{Chemistry}}\\
\end{center}

\begin{enumerate}
 \item \large When a metal is burnt, its weight is increased by 24 percent. The Equivalent weight of metal will be\\

		a.) 2 \hspace{2.5cm} b.) 24 \hspace{2.5cm} c.) 33.3 \hspace{2.5cm} d.) 76\\

\item \large In which one of the following, the number of protons is greater than number of neutrons but number of protons is less than the number of electrons?\\

		a.) $D_2O^{(+)}$ \hspace{2.2cm} b.) $SO_2$ \hspace{2.2cm} c.) $S^{2-}$ \hspace{2.2cm} d.) $OH^{-}$\\

\item \large The correct order of decreasing dipole moment of \\
		(I) toluene \hspace{6.1cm} (II) m-dichlorobenzene\\
		(III) o-dichlorobenzene \hspace{3.9cm} (IV) p-dichlorobenzene\\

		a.) $IV<II<I<III$ \hspace{1cm} b.) $IV<I<II<III$ \hspace{1cm} c.) $I<IV<II<III$
\hspace{1cm} d.) $IV<I<III<II$\\

\item \large The latent heats of fusion in $Jg^{-1}$ of five substances A (mol. mass=18) B(mol. mass=20), C(mol. mass=30), D(mol. mass=60) and E(mol. mass=30) are respectively 80, 45, 90, 45, 45. Which of the following pairs has same value of "$\Delta H_{fusion}$"?\\

		a.) A, C \hspace{2.5cm} b.) B, E \hspace{2.5cm} c.) D, E \hspace{2.5cm} d.) C, D\\

\item \large Which of the following is not formed as an intermediate in the Reimer-Teimann reaction between phenol and alkaline chloroform?\\

		a.)  \chemfig{*6(-=-(-CCl_2)-(=O)-=)}  b.) \chemfig{*6(--=-(=O)-=)} \qquad c.) \chemfig{*6(-=(-CHCl_2)-=(-OH)-=)}  d.) $:CCl^{2-}_2$\\

\item \large Which of the following statement(s) is/are true?\\
	
		a.) At room temperature, formyl chloride is present in the form of CO and HCl\\
		b.) Acetamide behaves as a weak base as well as a weak acid\\
		c.) \ce{CH_3CONH_2 ->[LiAlH_4] CH_3CH_2NH_2}\\
		d.) All the above are true
%---------------------------------------------------------------------------------------------------------------
\end{enumerate}

\newpage
\begin{center}
 \textbf{\large \underline{Physics}}\\
\end{center}

\begin{enumerate}
 \item \large In the circuit shown in the figure, reading of voltmeter is $V_1$ when only $S_1$ is closed, reading of voltmeter is $V_2$ when only $S_2$ is closed and reading of voltmeter is $V_3$ when both $S_1$ and $S_2$ are closed, then\\
	\begin{circuitikz}[scale=1.5]
 \draw
(0,2.5)  to[battery, V=E]  (0,5)
  to[R=R] (2,5) -- (4,5) -- (4,6) to[R=3R] (5,6) to[switch=${\rm s_1}$] (6,6) -- (6,5) --(7,5) to[voltmeter, l=$V_2$] (7,2.5) -- (0,2.5)
(4,5) -- (4,4) to[R=6R] (5,4) to[switch=${\rm s_2}$] (6,4) -- (6,5)

(3,5) -- (3,3) -- (3.5,3) to[voltmeter, l= $V_1$] (4.5,3) -- (6.3,3) -- (6.3, 5) 
;
	\end{circuitikz}

		a.) $V_3>V_2>V_1$ \hspace{5 cm} b.) $V_2>V_1>V_3$\\
		c.) $V_3>V_1>V_2$ \hspace{5 cm} d.) $V_1>V_2>V_3$\\ 

\end{enumerate}
%-------------------------------------------------------------------------------------------------------------------------------
\newpage
\begin{center}
\textbf{\large \underline{Biology}}\\
\end{center}

\begin{enumerate}
\item \large Diffusion pressure is directly proportional to:\\

		a.) concentration of molecules diffusing \\ b.) kinetic energy of diffusion molecules\\
		c.) concentration gradient\\  d.) all of the above\\

\item \large What happens when a formalin preserved filament of Spirogyra is placed in a hypertonic sugar solution?\\

		a.) it losses turgidity \\ b.) it gains turgidity \\ c.) it is plasmolysed \\ d.) nothing happens\\

\item \large Nif genes occur in\\

		a.) rhizobiium \\ b.) Aspergillus \\ c.) Penicillium \\ d.) Steptococcus\\

\item \large Gibberellin induces flowering in\\

		a.) some plants only\\
		b.) in long day plants under short day conditions\\
		c.) in short day plants under long day conditions\\
		d.) day neutral plants\\

\item \large $\alpha - 1$ Antitrypsin is\\
		a.) an antacid \\ b.) an enzyme \\ c.) used to treat arthritis \\ d.) used to treat emphysema

\end{enumerate}
\end{document}